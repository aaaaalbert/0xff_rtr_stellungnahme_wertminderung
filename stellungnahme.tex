% Mitgliedsantrag für FunkFeuer Wien, 2019
%
% Basierend auf den LaTeX-Templates für Verträge von housing.funkfeuer.at
% sowie dem FunkFeuer-Mitgliedsantragsformular von 2013

\documentclass[parskip=half]{scrreprt}

\usepackage{pdfpages}

\usepackage[utf8]{inputenc}
\usepackage[T1]{fontenc}
\usepackage[ngerman]{babel}
\usepackage{lmodern}
\usepackage[juratotoc]{scrjura}
%\usepackage[margin=2cm]{geometry}
%\usepackage{geometry}
% \usepackage[textheight=8in,footskip=60pt,bottom=1in]{geometry}
\usepackage[a4paper,margin=1.5cm,top=1.6cm,bottom=4cm]{geometry}
\usepackage{lastpage}
\usepackage{fancyhdr}% headers
\usepackage{graphicx} % logo include
\usepackage{color} % text colors
\usepackage{setspace}

\usepackage{paratype} % font
\renewcommand*\familydefault{\sfdefault} %% Only if the base font of the document is to be sans serif

\usepackage{amssymb}
\usepackage{url}

% FunkFeuer Colors
\definecolor{ff-dark}{RGB}{39,49,66}
\definecolor{ff-grey}{RGB}{85,98,112}
\definecolor{ff-light}{RGB}{105,145,170}
\definecolor{ff-red}{RGB}{242,58,20}
\definecolor{ff-green}{RGB}{161,200,32}


\usepackage{wrapfig}

% header
\pagestyle{fancy}
\fancyhf{}
\fancyhead[R]{\includegraphics[width=3.9cm]{funkfeuer_wien.pdf}}
%\fancyhead[L]{\textcolor{ff-grey}{\textbf{Verein zur Förderung freier Netze} - ZVR: 814804682} \\
%	\textcolor{ff-grey}{Postfach 44, 1016 Wien, Österreich}}
%\fancyhead[L]{\textcolor{ff-grey}{\textbf{Verein zur Förderung freier Netze}\\ZVR: 814804682}
\fancyhead[L]{\textcolor{ff-grey}{\textbf{Verein zur Förderung freier Netze}\\\small vorstand@funkfeuer.at}
}

% footer
%\fancyfoot[L]{\vskip -10pt~\includegraphics[width=1.5cm]{logo.pdf}}
% \fancyfoot[C]{\textcolor{ff-dark}{FunkFeuer Wien - ZVR: 814804682 \\Postfach 5  ·  1096 Wien \\ IBAN: AT552023000000143982  ·  BIC/SWIFT: SPLSAT21}}
\fancyfoot[L]{
\begin{wrapfigure}{L}{1.6cm}
\vspace{-15pt}
\includegraphics[width=1.75cm]{logo.pdf}
\end{wrapfigure}
\vspace{-5pt}
\fontsize{7}{8} \selectfont
\textcolor{ff-grey}{\hspace*{-1mm}Postanschrift:\\FunkFeuer Wien\\ Postfach 44\\ 1016 Wien}
}
\fancyfoot[C]{\fontsize{7}{8} \selectfont \textcolor{ff-grey}{\\ZVR: 814804682\\UID: ATU67830859\\BIC/SWIFT: SPLSAT21\\ IBAN: AT552023000000143982}
}
\fancyfoot[R]{\textcolor{ff-grey}{\thepage/\pageref{LastPage}}}

% lines
\renewcommand{\headrulewidth}{2pt}
\renewcommand{\headrule}{\hbox to\headwidth{%
  \color{ff-green}\leaders\hrule height \headrulewidth\hfill}}
\renewcommand{\footrulewidth}{0.5pt}
\renewcommand{\footrule}{\hbox to\headwidth{%
  \color{ff-light}\leaders\hrule height \footrulewidth\hfill}}
% section colors
\usepackage{sectsty} % autocolor sections
\chapterfont{\color{ff-grey}}  % sets colour of chapters
\sectionfont{\color{ff-grey}}  % sets colour of sections
% Text color
\color{ff-dark}



\begin{document}
\section*{\\Stellungnahme des Vereins FunkFeuer Wien zur RTR RVON 5/2018}
\thispagestyle{fancy}

Sehr geehrte Damen und Herren von der RTR!

Der Verein FunkFeuer Wien, Verein zur Förderung freier Netze
(ZVR 814804682), nimmt hiermit Stellung zur Konsultation
RVON 5/2018 bezüglich Richtsätze für Wertminderung durch
Antennentragemasten und Leitungsrechte.


\subsection*{Stellungnahme zur Vorlage im Detail}

\begin{description}

\item[§2, Zielgruppe der Vorlage.] Die Vorlage zielt dem TKG 2003
entsprechend auf Bereitsteller öffentlicher Kommunikationsnetze ab.
Wir regen an, die angepeilte Zielgruppe insofern zu erweitern, als dass
sowohl im nicht-kommerziellen (z.B. Vereins-) wie auch im privaten
Bereich ein Interesse an der geregelten Nutzung von Antennenstandorten
besteht. Dieses ist besonders in Gebieten anzutreffen, wo kommerzielle
Betreiber kein technisches Portfolio, keine ökonomischen Anreize
oder kein strategisches Interesse an einer Versorgung haben.
\textbf{Daraus leiten wir den Wunsch ab, explizite, angepasste
Kostensätze für nichtkommerzielle Nutzung und Regeln für deren
Verfügbarkeit zu definieren.}

\item[Wunsch: Rahmenverträge.] Für eine effiziente Nutzbarmachung von
Antennenstandorten sehen wir es weiters als sinnvoll an, seitens der
RTR ein Rahmenvertragswerk anzubieten, das die notwendigen Vereinbarungen
kodifiziert und die Notwendigkeit separater bilateraler (aber insgesamt
inhaltlich jeweils annähernd identischer) Vertragsverhandlungen
zwischen den den Standort bzw. den Infrastrukturaufbau vertretenden
Parteien minimiert.

\item[§1, Mögliche Redundanzen zum TKG 2003.] Einige Definitionen in der
Vorlage scheinen sich mit Begriffsbestimmungen im TKG 2003 zu
überschneiden bzw. diese inhaltlich, aber unter einem neuem Begriff zu
doppeln. Beispielsweise definiert §1 der Vorlage ,,Inhouse-Infrastruktur'',
im Vergleich zum TKG §3 ,,gebäudeinterne physische Infrastruktur''.

\item[§8, Umfang des Richtsatzes 4.] Wird mit dem einmaligen Betrag
nach Richtsatz eine Antenne, ein Mast, ein Gebäude, ein Grundstück,
etwas anderes abgegolten?

\item[Pauschalisierung.] Die Berechnungen zum
,,Beispiel 1'' in den
Erläuterungen ergeben eine Wertminderung von 13 Euro 65 für eine
Künette von 30 Laufmetern Länge. Beträge wie diese überhaupt zu
fakturieren und zu bezahlen verursacht mehr Aufwand, als es Umsatz
bringt. Insofern wäre eine Pauschalisierung (d.h. ein jedenfalls zu
entrichtender Mindestbetrag, falls sich eine Wertminderung unter
einer gewissen Schwelle ergibt) oder sogar eine Bagatellgrenze
(d.h. ein Betrag, unterhalb dessen überhaupt keine Wertminderung
abzugelten wäre) überlegenswert. Einer absichtlichen ,,Stückelungen''
von Abgeltungen könnte einerseits durch entsprechende zeitliche oder
funktionale Gruppierung in der Bewertung vorgegriffen werden,
andererseits ließe sich auch der Kreis der Nutznießer\_innen für
eine Pauschal- oder Bagatellregelung einschränken, beispielsweise
auf die erwähnten Privatpersonen oder Vereine.



\end{description}





\subsection*{Stellungnahme zu den Erläuterungen im Detail}

\begin{description}

\item[Zielgruppe.] Die Erläuterungen erwähnen als explizites Ziel
der Vorlage die ,,Förderung der Standortqualität''. Wir unterstreichen,
dass zu dieser Förderung neben den in der Vorlage angesprochenen
kommerziellen Betreibern in maßgeblichem Umfang auch nichtkommerziell
orientierte Initiativen, Privatpersonen, Vereine, Funkamateur\_innen
usw. beitragen.

\item[Notwendigkeit der Erläuterungen.] Ohne die von der RTR im
Rahmen der Konsultationsunterlagen zur Verfügung gestellten
,,Erläuterung'' ist ein Verständnis der Inhalte, insbesondere aber
auch der Motivationen der konkreten Ausgestaltung der Vorlage,
schwierig bis unmöglich. Das gilt beispielsweise für die
Erläuterungen zum Flächenbedarf in §7 oder das Zustandekommen
der angeführten Richtsätze. 

\item[Einschränkung auf Bauland und Grünland.] Wie der Abschnitt
,,Begriffsbestimmungen'' im \textit{Besonderen Teil} der Erläuterungen
zur Vorlage festhält, werden für Flächen und Widmungen abseits von
Bauland und Grünland keine Richtsätze vorgeschlagen. Gleichzeitig
erkennt die Erläuterung die Existenz anderer Flächen- und Widmungstypen
an. Wir erachten eine Erweiterung der Richtsätze unter der oben genannten
Prämisse der Erweiterung der Zielgruppe auf Privatpersonen, Vereine
usw. als notwendig, zumal Kommunikationsinfrastruktur bereits heutzutage
in großem Ausmaß entlang von Straßen, Bahnstrecken, Gas-, Strom- und Fernwärmeleitungen etc. geführt wird.

\item[Einschränkung der Verbindlichkeit.] Auf Seite 2 (mitte) der
Erläuterungen wird eingeschränkt, dass die Richtsätze im Vorschlag
gerade \textit{keine} verbindliche Anordnung seien. Wir bestreiten
keinesfalls die Konsistenz dieser Einschränkung mit dem TKG 2003,
bemerken aber auch, dass der Verordnung von Richtsätzen insofern
ein allermeistens nur informativer Charakter zukommen wird und nicht
nur kaum grundsätzliche Vereinfachungen von Vertragsverhandlungen
zwischen Beteiligten schaffen wird, sondern vielmehr geradezu eine
Einladung darstellt, die Regulierungsbehörde in verstärktem Maße
in per se triviale und unumstrittene Verfahren zu involvieren.

\item[Richtsatzbildung.] Die Diskussion der Richtsatzbildung ist
dankenswerterweise sehr offen. Wir kritisieren allerdings die
Verwendung kommerziell gesammelter Daten für Richtsatz 1; während
wir dessen Open-Data-Qualität ausdrücklich anerkennen und die
beschriebene Methodik (Medianpreise aus Grundbuchtransaktionen)
plausibel erscheint, würden wir die direkte und ausschließliche
Nutzung statistischer Daten der öffentlichen Hand empfehlen.
Das würde zusätzlich die Flexibilität in der Bewertung stark
divergierender Werte pro Region ermöglichen, eine Divergenz,
die sowohl in der Beschreibung der IMMOunited-Datenquelle z.B.
für Wien (,,heterogene Preislage'') genannt wird, als auch der
oben besprochenen Einschränkung der Verbindlichkeit der vorgeschlagenen
Verordnung gemäß TKG zugrunde zu liegen scheint.

\end{description}

\vspace{1.5cm}

Für den Verein FunkFeuer Wien

Vorname Nachname (Funktion)


\end{document}
